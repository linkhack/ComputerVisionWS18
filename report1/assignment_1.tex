\section{Assignment 1: Colorizing Images}
\label{sec:assignment1}

The goal of this assignment is to assemble a colorized image given three color channel images. As the three channels are not perfectly aligned the program has to also align them automatically.


\subsection{Methodology}

For the above purpose, we use normalized cross-correlation as the matching metric. The return value of the metric for each displacement (-OFFSET:OFFSET) is saved and the best displacement is used to create the final image.
In the initial version, we compare the values of Red channel as the base with Green and Blue Channels. The best displacement of Green and Blue compared to Red are used in the final image.
In a second version, first the best displacement for Green channel against Red channel is determined. Then we compare various Blue channel displacements against Red+Green2 (displacement of green channel found in previous step). The Blue displacement with the best score sum is used in the final channel.
The results were fairly equal in both cases, however the second experiment appears to deliver better alignment slightly.




